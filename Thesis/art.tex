\chapter{State of Art}
\section{Previous Researches}
To show the novelty of my work, at first need to understand what was researched previously.

Same stages of the actions are used to construct a phylogenetic tree in many methods.
They can be divided in three main actions.
It is necessary to align sequences so that able to compare each other. 
One of the earliest formalizations of multiple sequence when it as the 
problem of reconstructing ancient predecessors to contemporary sequences 
when the topology of a tree describing the evolution of the sequences is given.
The edit distances between the sequences at the nodes of this tree define the 
lengths of the edges. The problem is to choose the ancient sequence such that 
the overall length of th tree is minimized. This formulation is called 
``tree-alligment''. It is one of two commonly used formulations of multiple 
sequence alignment, the other one being the so-called sum-of-pairs alignment.

\subsection{Multiple Sequence Alignment}
Alignment methods vary in type of data (DNA, RNA, or amino-acid)
they can handle, and also, to some extent, the objectives of the alignment method \cite{tandy}.
Thus, some methods are designed exclusively for proteins, some exclusively
for RNAs, but many alignment methods can analyze both protein and nucleotide
datasets. We refer to methods that can analyze all types of molecular sequences as
``generic'' methods.

Alignment methods are used to predict function and structure, to determine whether a
sequence belongs to a particular gene family or superfamily, to recognize homology in the
``twilight zone'' (where sequence similarity is so low that homology is dificult to detect),
to infer selection, etc.. On the other hand, alignment methods are also used in order
to estimate a phylogeny. As we shall see, the design of alignment methods and how they
are evaluated depend on the purpose they are being used for.


\subsubsection{MSA Evaluation Criteria}
The standard criteria used to evaluate alignments for
accuracy are based on shared homologies between the true and the estimated alignment,
with the SP-score (sum-of-pairs score) measuring the fraction of the true pairwise
homologies correctly recovered, and the TC-score (``total column'' score) measuring the
number of identical columns. Variants on these criteria include the true metrics suggested
by Blackburne et al. and the consideration of diferent types of alignment error (i.e.,
both false positive and false negative) rather than one overall measure of ``alignment
accuracy''. Other criteria, such as the identification and correct alignment of specific
regions within a protein or rRNA, have also been used. Furthermore, because
sequence alignment has the potential to impact phylogeny estimation, a third way of
evaluating a multiple sequence alignment method is via its impact on phylogeny estimation

\subsubsection{Benchmark datasets}
Many studies have
evaluated MSA methods using biological data for which structurally informed alignments
are available. The best known of these benchmark datasets is probably BAliBASE,
but others are also used. The choice of benchmarks and how they are used
has a large impact on the result of the evaluation, and so has been discussed in several
papers.

From the perspective of phylogeny estimation, one of the
most important of their observations is that structurally-defined benchmarks often omit
the highly variable parts of the molecule, including introns. Thus, an alignment can be
considered completely correct as a structural alignment if it aligns the conserved regions,
even if it fails to correctly align the variable regions. The problem with this criterion (as
they point out) is that the highly variable portions are often the sites that are of most use
for phylogeny estimation, whereas sites that change slowly have little phylogenetic signal.

\subsubsection{Relative performance of MSA methods}
While most studies have evaluated alignment methods in terms of standard criteria 
(notably, SP and TC scores) on biological
benchmarks, some studies have explored alignment estimation for phylogeny estimation
purposes. As commented on earlier, many studies have shown that alignment estimation
impacts phylogenetic estimation, and that alignment and tree error increase with the rate
of evolution. Also, on very large datasets, due to computational limitations, only a few
alignment methods can even be run (and typically not the most accurate ones), which
results in increased alignment error. On the other hand, on large trees with rates of
evolution that are sufficiently low, alignment estimation methods can differ substantially
in terms of SP-score without impacting the accuracy of the phylogenetic tree estimated
using the alignment. More generally, standard alignment metrics may be only poorly
correlated with tree accuracy in some conditions.

\subsubsection{Progressive aligners, and the impact of guide trees}
Many alignment methods
use progressive alignment on a guide tree to estimate the alignment; thus the choice of the
guide tree and its impact on alignment and phylogeny estimation is also of interest. 
Nelesen studied the impact of the guide tree on alignment methods,
and showed that improved phylogenetic accuracy can be obtained by first estimating a tree
from the input using a good two-phase method (RAxML on a MAFFT alignment).
They noted particular benefits in using Probcons with this guide tree, and called the
resultant method ``Probtree''. Prank has also been observed to be very sensitive to guide
trees, and to give improved results by the use of carefully computed guide trees
(maximum likelihood on good alignments). Another study showed that even
when the alignment score doesn't change, the alignment itself can change in important
ways when guide trees are changed. Finally, Capela-Gutierrez and Gabaldon
found that the placement of gaps in an alignment results from the choice of the guide tree,
and hence the gaps are not phylogenetically informative. Based on these observations,
Capela-Gutierrez and Gabaldon recommended that alignment estimation methods should
use the true tree (if possible), or else use an iterative co-estimation method that infers
both the tree and the alignment.

\subsubsection{Template-based methods}
Some alignment methods use a very different type of
algorithmic structure, which is referred to as being ``template-based''. Instead of
using progressive alignment on a guide tree, these methods use models (either profiles,
templates, or Hidden Markov Models) for the gene of interest, and align each sequence to
the model in order to produce the final multiple sequence alignment, as follows. First, a
relatively small set of sequences from the family is assembled, and an alignment estimated
for the set. Then, some kind of model (e.g., a template or a Hidden Markov Model) is
constructed from this ``seed'' alignment. This model can be relatively simple or quite
complex, typically depending on whether the model provides structural information. Once
the model is estimated, the remaining sequences are added to the growing alignment. The
model is used to align each sequence to the seed alignment (which does not change during
the process), and then inserted into the growing alignment. Since the remaining sequences
are only compared to the seed alignment, homologies between the remaining sequences
can only be inferred through their homologies to the seed alignment. Thus, the choice
of sequences in the seed alignment and how it is estimated can have a big impact on
the resultant alignment accuracy. By design, once the seed alignment and the model
are computed, the running time scales linearly with the number of sequences, and the
algorithm is trivially parallelizable. Thus, these methods, which we will refer to jointly as
``template-based methods'', can scale to very large datasets with hundreds of thousands
of sequences.

There are several examples of methods that use this approach. Some of these methods use curated seed alignments based on
structure and function of well-characterized proteins or rRNAs; for example, the protein
alignment method by Neuwald and the rRNA sequence alignment method by Gardner 
use curated alignments. Constraint-based methods, such as COBALT,
3DCoffee and PROMALS, similarly use external information like structure and
function, but then use progressive alignment techniques (or other such methods) to 
produce the final alignment. Clustal-Omega also has a version, called ``External Profile
Alignment'', that uses external information (in the form of alignments) to improve the
alignment step.

Finally, PAGAN is another member of this class of methods; however, it has
some specific methodological differences to the others. First, unlike several of the others,
it does not use external biological information (about structure, function, etc.) to define
its seed alignment. Second, while the others tend to use either HMMs, profiles, or 
templates as a model to define the alignment of the remaining sequences, PAGAN estimates
a tree on its seed alignment, and estimates sequences for the internal nodes. These 
sequences are then used to define the incorporation of the remaining sequences to the seed
alignment. This technique is very similar to the technique used in PaPaRa, which
was developed for the phylogenetic placement problem. Thus, PAGAN
is one of the ``phylogeny-aware'' alignment methods, a technique that is atypical of these
template-based methods, but shared by progressive aligners. PAGAN was compared to an
HMM-based method (using HMMER on the reference alignment to build an HMM, and
then using HMMALIGN to align the sequences to the HMM) on several datasets.
The comparison showed that PAGAN had very good accuracy, better than HMMALIGN,
under low rates of evolution, and that both methods had reduced accuracy under high
rates of evolution. They also noted that PAGAN failed to align some sequences under
model conditions with high rates of evolution, while HMMER aligned all sequences; 
however, the sequences that both HMMER and PAGAN aligned were aligned more accurately
using PAGAN.

\subsubsection{Methods that use divide-and-conquer on the taxon set}
Some alignment methods use a divide-and-conquer strategy in which the 
taxon set is divided into subsets
(rather than the sites) in order to estimate the alignment; these include the 
mega-phylogeny method developed by Smith, SAT'e, SATCHMO-JS,
PROMALS, and the method by Neuwald. (The SAT'e and SATCHMO-JS 
methods co-estimate alignments and trees, and so are not strictly speaking just alignment
methods.) Neuwald's method is a bit of an outlier in this set, because the user provides
the dataset decomposition, but we include it here for comparative purposes.

While the methods differ in some details, they use similar strategies to estimate 
alignments. Most estimate an initial tree, and then use the tree to divide the dataset into
subsets. The method to compute the initial trees differs, with SATCHMO-JS using a
neighbor joining (NJ) tree on a MAFFT alignment, SAT'e using a maximum 
likelihood tree on a MAFFT alignment, PROMALS using a UPGMA tree on $k$-mer distances,
and mega-phylogeny using a reference tree and estimated alignment.

The subsequent division into subsets is performed in two ways. In the case of 
mega-phylogeny, SATCHMO-JS, and PROMALS, the division into subsets is performed by
breaking the starting tree into clades so as to limit the maximum dissimilarity between
pairs of sequences in each set. In contrast, SAT'e-2 removes centroid edges from the
unrooted tree, recursively, until each subset is small enough (below 200 sequences). Thus,
the sets produced by the SAT'e-2 decomposition do not form clades in the tree, unlike the
other decompositions. Furthermore, the sets produced by the SAT'e-2 decomposition are
guaranteed to be small (at most 200 taxa) but are not constrained to have low pairwise
dissimilarities between sequences.

Alignments are then produced on each subset, with PROMALS, SAT'e, and 
mega-phylogeny estimating alignments on each subset, and SATCHMO-JS using the alignment
induced on the subset by the initial MAFFT alignment.

These alignments are then merged together into an alignment on the full set, but the
methods use different techniques. PROMALS and mega-phylogeny use template-based
methods to merge the alignments together, while SATCHMO-JS and SAT'e use progressive
alignment techniques. PROMALS also uses external knowledge about protein structure
to guide the template-based merger of the alignments together. PROMALS, 
SATCHMO-JS, and mega-phylogeny use sophisticated methods to merge subset-alignments, but SAT'e
uses a very simple method (Muscle) to merge subset-alignments.

\subsubsection{Very large-scale alignment}
When the datasets are very large, containing many
thousands of sequences, only a few alignment estimation methods are able to run. As
noted, the template-based methods (including PROMALS and mega-phylogeny) scale
linearly with the number of taxa, and so can be used with very large datasets. SAT'e
and SATCHMO-JS are not quite as scalable; however, SAT'e has been able to analyze
nucleotide datasets with about 28,000 sequences. Other methods that have been shown to
run on very large datasets include Clustal-Omega, MAFFT-PartTree, and 
Kalign-2, but many methods fail to run on datasets with tens of thousands of sequences.
Of these, Clustal-Omega is only designed for protein sequences, but MAFFT-PartTree
and Kalign-2 can analyze both nucleotide and amino-acid sequences.

SAT'e is computationally limited by its use of progressive alignment and maximum
likelihood method (RAxML or FastTree-2) in each iteration; both impact the 
running time and - in the case of large numbers of long sequences - memory usage. However,
although limited to datasets with perhaps only 30,000 sequences (or so), on fast-evolving
datasets with 1000 or more sequences, SAT'e provides improvements in phylogenetic 
accuracy relative to competing methods.

\subsection{Stochastic models of sequence evolution}
We begin with a description of the simplest stochastic models of DNA sequence 
evolution, and then discuss amino-acid sequence evolution models and codon evolution models.
The simplest models of DNA sequence evolution treat the sites within the sequences 
independently. Thus, a model of DNA sequence evolution must describe the probability
distribution of the four states, A, C, T, G, at the root, the evolution of a random site (i.e.,
position within the DNA sequence) and how the evolution differs across the sites. 
Typically the probability distribution at the root is uniform (so that all sequences of a fixed
length are equally likely). The evolution of a single site is modeled through the use of
``stochastic substitution matrices'', $4 x 4$ matrices (one for each tree edge) in which every
row sums to 1. A stochastic model of how a single site evolves can thus have up to 12 free
parameters. The simplest such model is the Jukes-Cantor model, with one free parameter,
and the most complex is the General Markov model, with all 12 parameters:

\begin{defn}
The General Markov (GM) model of single-site evolution is defined as follows.

\begin{enumerate}
	\item The nucleotide in a random site at the root is drawn from a known distribution, in
which each nucleotide has positive probability.
	\item The probability of each site substitution on an edge $e$ of the tree is given by a $4 x 4$
stochastic substitution matrix $M(e)$ in which $det(M(e))$ is not $0$, $1$, or $-1$.	
\end{enumerate}
\end{defn}

This model is generally used in a context where all sites evolve identically and 
indpendently (the i.i.d. assumption), with rates of evolution drawn typically drawn from a
gamma distribution. (Note that the distribution of the rates-across-sites has an impact
on phylogeny estimation and dating, as discussed by Evans and Warnow.) In what
follows, we will address the simplest version of the GM model so that all sites have the
same rate of evolution.

We denote a model tree in the GM model as a pair, ($T, {M(e): e \in E(T)}$), or more
simply as $(T, M)$. For each edge $e \in E(T)$, we define the length of the edge $\lambda(e)$ to
be - $log |det(M(e))|$. This allows us to define the matrix of leaf-to-leaf distances, ${\lambda_{ij}}$,
where $\lambda_{ij} = \sum_{e \in P_{ij}}$
$\lambda(e)$, and where $P_{ij}$ is the path in $T$ between leaves $i$ and $j$. A
matrix defined by path distances in a tree with edge weights is called ``additive'', and it
is a well-known fact that given any additive matrix, it is easy to recover the underlying
leaf-labelled tree T for that matrix in polynomial time.

This general model of site evolution subsumes the great majority of other models
examined in the phylogenetic literature, including the popular General Time Reversible
(GTR) model, which requires only that $M(e) = M(e')$
for all edges $e$ and $e'$. 
Further constraints on the matrix $M(e)$ produce the Hasegawa-Kishino-Yang (HKY) model,
the Kimura 2-parameter model (K2P), the Kimura 3-ST model (K3ST), the Jukes-Cantor
model (JC), etc. These models are all special cases of the General Markov model, because
they place restrictions on the form of the stochastic substitution matrices. The standard
model used for nucleotide phylogeny estimation is GTR+gamma, i.e., the General Time
Reversible (GTR) model of site substitution, equipped with a gamma distribution of rates
across sites.

\subsubsection{Protein models}
Just as with DNA sequence evolution models, there are Markov
models of evolution for amino-acid sequences, and also for coding DNA sequences. These
models are described in the same way - a substitution matrix that governs the tree,
and then branch lengths. While the GTR model can be extended to amino-acids (to
produce a $20x20$ matrix) or to codon-based models (to produce a $64x64$ matrix), both
of which must be estimated from the data, in practice these models use fixed matrices,
each of which was estimated from external biological data. The most well known protein
model is the Dayhoff model, but improved models have been developed in recent
years. Similarly, codon-based models have also been based on fixed $64x64$
matrices. In practice, the selection of a protein model for a given dataset
is often done using a statistical test, such as ProtTest, and then fixed. In the
subsequent tree estimation performed under that model, only the tree and its branch
lengths need to be estimated.

\subsection{Phylogeny Estimation Methods}
There are many different phylogeny estimation methods, too numerous to mention here.
However, the major ones can be classified into the following types:

\begin{itemize}
	\item distance-based methods, which first compute a pairwise distance matrix (usually
based on a statistical model) and then compute the tree from the matrix,
	\item maximum parsimony and its variants, which seek a tree with a total minimum
number of changes (as defined by edit distances between sequences at the endpoints
on the edges of the tree),
	\item maximum likelihood, which seeks the model tree that optimizes likelihood
under the given Markov model, and
	\item Bayesian MCMC methods, which return a distribution on trees rather than a single
tree, and also use likelihood to evaluate a model tree.	
\end{itemize}

\subsection{Statistical Performance Criteria}
We discuss three concepts here: \textit{identifiability, statistical consistency, 
and sequence length requirements}.


\textbf{Identifiability:} A statistical model or one of its parameters is said to be 
``identifiable'' if it is uniquely determined by the probability distribution 
defined by the model. Thus, in the context of phylogeny estimation, the 
unrooted model tree topology is identifiable if it is determined by the 
probability distribution (defined by the model tree, which includes the 
numeric parameters) on the patterns of nucleotides at the leaves of the tree. 
In the case of nucleotide models, the state at each leaf can be A,C,T, or G, 
and so there are $4^n$ possible patterns in a tree with n leaves 
(similarly, there are $20^n$ possible patterns for amino-acid models). 
It is well known that the unrooted tree topology is identifiable under 
the General Markov model, and recent work has extended this to 
other models.

\textbf{Statistical Consistency:} We say that a method $\Phi$ is ``statistically consistent'' 
for estimating the topology of the model tree $(T,\theta )$ if the trees estimated 
by $\Phi$ converges to the unrooted version of $T$ (denoted by $T^u$) as the number 
of sites increases. (Note that under this definition, we are not concerned 
with estimating the numeric parameters.) Equivalently, for all $\varepsilon > 0$ there 
is a sequence length $K$ so that if a set $S$ of sequences of length $k \geq K$ are 
generated by $(T,\theta)$, then the probability that $\Phi(S) = T^u$ is at least $1-\varepsilon$. 
We say that a method is statistically consistent under the GM model if 
it is statistically consistent for all model trees in the GM model. 
Similarly, we say a method is statistically consistent under the GTR 
model if it is statistically consistent under all model trees in the GTR model. 

Many phylogenetic methods are statistically consistent under the GM model, 
and hence also under its submodels (e.g., the GTR model). For example, 
maximum likelihood, neighbor joining (and other distance-based methods) 
for properly computed pairwise ``distances'', and Bayesian MCMC methods, 
are all statistically consistent. Onthe other hand, maximum parsimony and 
maximum compatibility are not statistically consistent under the GM 
model. In addition, it is well known that maximum likelihood can be 
inconsistent if the generative model is different from the model 
assumed by maximum likelihood, but maximum likelihood can even be 
inconsistent when its assumptions match the generative model, 
if the generative model is too complex! For example, 
Tuffey and Steel showed that maximum likelihood is equivalent 
to maximum parsimony under a very general ``no-common-mechanism'' model, 
and so is inconsistent under this model. In this case, the model 
itself is not identifiable, and this is why maximum likelihood 
is not consistent. However, there are identifiable models for 
which ML is not consistent, as observed by Steel.

\textbf{Sequence length requirement:} Clearly, statistical consistency 
under a model is a desirable property. However, statistical consistency 
does not address how well a method will work on finite data. 
Here, we address the ``sequence length requirement'' of a phylogeny 
estimation method $\Phi$, which is the number of sites that $\Phi$ needs 
to return the (unrooted version of the) true tree with 
probability at least $1 - \epsilon$ given sequences that evolve 
down a given model tree $(T,\theta)$. Clearly, the number of 
sites that suffces for accuracy with probability at 
least $1 - \epsilon$ will depend on $\Phi$ and $\epsilon$, but it also 
depends on both $T$ and $\theta$. 

We describe this concept in terms of the Jukes-Cantor model, 
since this is the simplest of the DNA sequence evolution models, 
and the ideas are easiest to understand for this model. However, 
the same concepts can be applied to the more general models, and 
the theoretical results that have been established regarding sequence 
length requirements extend to the GM (General Markov) model, which 
contains the GTR model and all its submodels.

In the Jukes-Cantor (JC) model, all substitutions are equally likely, 
and all nucleotides have equal probability for the root state. 
Thus, a Jukes-Cantor model tree is completely defined by the rooted 
tree $T$ and the branch lengths $\lambda(e)$, where $\lambda(e)$ is 
the expected number of changes for a random site on the edge $e$. 
It is intuitively obvious that as the minimum branch length 
shrinks, the number of sites that are needed to reconstruct the 
tree will grow, since a branch on which no changes occur 
cannot be recovered with high probability (the branch will appear 
in an estimated tree with probability at most one-third, 
since at best it can result from a random resolution of a 
node of degree at least 4). It is also intuitively obvious 
that as the maximum branch length increases, the number of 
sites that are needed will increase, since the two sides 
of the long branch will seem random with respect to each other. 
Thus, the sequence length requirement for a given method to be 
accurate with probability at least $1 - \epsilon$ will be 
impacted by the shortest branch length f and the longest 
branch length $g$. It is also intuitively obvious that the 
sequence length requirement will depend on the number of 
taxa in the tree.  

Expressing the sequence length requirement for the 
method $\Phi$ as a function of these parameters ($f,g,n$ and $\epsilon$) 
enables a different - and finer - evaluation of the method's performance 
guarantees under the statistical model. Hence, we consider $f,g$, and $\epsilon$ 
as fixed but arbitrary, and we let JCf,g denote all Jukes-Cantor model 
trees with $0 < f \leq \lambda(e) \leq g < \infty$ for all edges $e$. 
This lets us bound the sequence length requirement of a method as a 
function only of $n$, the number of leaves in the tree. 

The definition of ``absolute fast convergence'' under the Jukes-Cantor 
model is formu- lated as an upper bound on the sequence length requirement, 
as follows:

\begin{defn}
A phylogenetic reconstruction method $\Phi$ is absolute fast-converging (afc) 
for the Jukes-Cantor (JC) model if, for all positive $f,g$, and $\varepsilon$, 
there is a polynomial $p(n)$ such that, for all $(T,\theta)$ in $JC_{f,g}$, 
on set $S$ of $n$ sequences of length at least $p(n)$ generated on $T$, 
we have $Pr[\Phi(S) = T^u] > 1 - \varepsilon$.
\end{defn}

Note also that this statement only refers to the estimation of the unrooted 
tree topology $T^u$ and not the numeric parameters $\Phi$. Also, note that 
the method $\Phi$ operates without any knowledge of parameters $f$ or $g$-or 
indeed any function of $f$ and $g$. Thus, although the polynomial $p$ 
depends upon both $f$ and $g$, the method itself will not. Finally, this 
is an upper bound on the sequence length requirement, and the actual 
sequence length requirement could be much lower. 

The function $p(n)$ can be replaced by a function $f(n)$ that is not 
polynomial to provide an upper bound on the sequence length requirement 
for methods that are not proven to be absolute fast converging. 

\subsection{Empirical Performance}
So far, these discussions have focused on theoretical guarantees under a model, 
and have addressed whether a method will converge to the true tree given 
long enough sequences (i.e., statistical consistency), and if so, then 
how long the sequences need to be (sequence length requirements). However, 
these issues are purely theoretical, and do not address how accurate the 
trees estimated by methods are in practice (i.e., on data). In addition, 
the computational performance (time and memory usage) of phylogeny estimation 
methods is also important, since a method that is highly accurate but will 
use several years of compute time will not generally be useful in most analyses. 

Phylogenetic tree accuracy can be computed in various ways, and there are 
substantive debates on the ``right'' way to calculate accuracy; however, 
although disputed, the Robinson-Foulds (RF) distance, also called the 
``bipartition distance'', is the most commonly used metric on phylogenetic trees. 
We describe this metric here. 

Given a phylogenetic tree $T$ on $n$ taxa, each edge can be associated 
with the bipartition it induces on the leaf set; hence, the tree 
itself can be identified with the set of leaf- bipartitions defined 
by the edges in the tree. Therefore, two trees on the same set 
of taxa can be compared with respect to their bipartition sets. 
The RF distance between two trees is the size of the symmetric difference 
of these two sets, i.e., it is the number of bipartitions that are in one 
tree's dataset but not both. This number can be divided by $2(n - 3)$ 
(where n is the number of taxa) to obtain the ``RF rate''. In the context 
of evaluating phylogeny estimation methods, the RF distance is sometimes 
divided into false negatives and false positives, where the false 
negatives (also called ``missing branches'') are branches in the 
true tree that are not present in the estimated tree, and the 
false positives are the branches in the estimated tree that are not 
present in the true tree. This distinction between false positives 
and false negatives enables a more detailed comparison between trees 
that are not binary.

Many studies have evaluated phylogeny estimation methods on simulated 
data, varying the rate of evolution, the branch lengths, the number 
of sites, etc. These studies have been enormously informative about 
the differences between methods, and have helped biologists make 
informed decisions regarding methods for their phylogenetic analyses. 
Some of the early simulation studies explored performance on very 
small trees, including the fairly exhaustive study by Huelsenbeck 
and Hillis on 4-leaf trees, but studies since then have explored 
larger datasets and more complex questions. For example, studies 
have explored the impact of taxon sampling on phylogenetic 
inference, the impact of missing data on phylogenetic inference, 
and the number of sites needed for accuracy with high probability. 
In fact, simulation studies have become, perhaps, the main way to 
explore phylogenetic estimation. 

\subsubsection{Distance-based methods} \label{methods}
Distance-based methods operate by first computing a matrix of 
distances (typically using a statistically defined technique, 
to correct for unseen changes) between every pair of sequences, 
and then construct the tree based on this matrix. Most, but not all, 
distance-based methods are statistically consistent, and so will 
be correct with high probability, given long enough sequences. 
In general, distance-based methods are polynomial time, and so 
have been popular for large-scale phylogeny estimation. 
While the best known distance-based method is probably neighbor 
joining, there are many others, and many are faster and/or 
more accurate. 

One of the interesting properties about distance-based 
methods is that although they are typically guaranteed 
to be statistically consistent, not all distance-based 
methods have good empirical performance! A prime example 
of this lesson is the Naive Quartet Method, a method that 
estimates a tree for every set of four leaves using the 
Four-Point Method (a statistically-consistent distance method) 
and then returns the tree that is consistent with all the 
quartets if it exists. It is easy to show that the Naive 
Quartet Method runs in polynomial time and is statistically 
consistent under the General Markov model; however, because 
it requires that every quartet be accurately estimated, 
it has terrible empirical performance! Thus, while 
statistical consistency is desirable, in many cases statistically 
inconsistent methods can outperform consistent ones.

\subsubsection{Maximum parsimony}
Maximum parsimony (MP) is NP-hard, and so the methods for 
MP use heuristics (most without any performance guarantees). 
The most efficient and accurate maximum parsimony software 
for very large datasets is probably TNT, but PAUP* is also 
popular and effective on datasets that are not extremely large. 
TNT is a particularly effective parsimony heuristic for 
large trees, and has been able to analyze a multi-marker 
sequence dataset with more than 73,000 sequences.

\subsubsection{Maximum likelihood}
Maximum likelihood (ML) is also NP-hard, and so attempts to 
solve ML are also made using heuristics. While the heuristics 
for MP used to be computationally more efficient than the 
heuristics for ML, the current set of methods for ML are 
quite effective at ``solving'' large datasets. (Here the 
quotes indicate that there is no guarantee, but reasonably 
good results do seem to be obtained using the current best software.)

The leading methods for large-scale ML estimation under the 
GTR+Gamma model include RAxML, FastTree-2, PhyML, and GARLI. 
Of these four methods, RAxML is clearly the most frequently 
used ML method, in part because of its excellent parallel 
implementations. However, a recent study showed that trees 
estimated by FastTree-2 were almost as accurate as those 
estimated by RAxML, and that FastTree-2 finished in a 
fraction of the time; for example, FastTree-2 was able to 
analyze an alignment with almost 28,000 rRNA sequences 
in about 5 hours, but RAxML took much longer. 
Furthermore, FastTree-2 has been used to analyze larger 
datasets (ones with more sequences) than RAxML: the 
largest dataset published with a RAxML analysis 
had 55,000 nucleotide sequences, but FastTree has 
analyzed larger datasets. For example, FastTree-2 
has analyzed a dataset with more than 1 million nucleotide 
sequences, and another with 330,556 sequences. 
The reported running time for these analyses are 203 
hours for the million-taxon dataset, and 13 hours 
(with 4 threads) for the 330K- taxon dataset 
\footnote{Morgan Price, personal communication, May 1, 2013.}. 
By comparison, the RAxML analysis of 55,000 nucleotide 
sequences took between 100,000 and 300,000 CPU hours
\footnote{Alexis Stamatakis, personal communication, May 1, 2013.}. 
The difference in running time is substantial, but we 
should note two things: the RAxML analysis was a 
multi-marker analysis, and so the sequences were much 
longer (which impacts running time), and because RAxML 
is highly parallelized, the impact of the increased running 
time is not as significant (if one has enough processors). 
Nevertheless, for maximum likelihood analysis of alignments 
with large numbers of sequences, FastTree-2 provides 
distinct speed advantages over RAxML.  

There are a few important limitations for FastTree-2, 
compared to RAxML. First, FastTree-2 obtains its speed by 
somewhat reducing the accuracy of the search; thus, the 
trees returned by FastTree-2 may not produce maximum 
likelihood scores that are quite as good as those produced 
by RAxML. Second, FastTree-2 doesn't handle very long 
alignments with hundreds of thousands of sites very well, 
while RAxML has a new implementation that is designed 
specifically for long alignments. Third, FastTree-2 has a 
smaller set of models for amino-acid analyses than RAxML. 
Therefore, in some cases (e.g., for wide alignments, 
and perhaps for amino-acid alignments), RAxML may 
be the preferred method. 

However, the ML methods discussed above estimate trees under 
the GTR+Gamma model, which has simplifying assumptions that 
are known to be violated in biological data. 
The nhPhyml method is a maximum likelihood method for 
estimating trees under the non-stationary, non-homogeneous 
model of Galtier and Guoy, and hence provides an 
analytical advantage in that it can be robust to some 
violations of the GTR+Gamma model assumptions. However, 
nhPhyml seems to be able to give reliably good analyses 
only on relatively small datasets (i.e., with at most a 
few hundred sequences, or fewer sequences if they are very long). 
The explanation is computational - it uses NNI 
(nearest neighbor interchanges, see below) to search treespace, 
but NNI is relatively ineffective, which means that it is 
likely to get stuck in local optima. This is unfortunate, 
since large datasets spanning substantial evolutionary 
distances are most likely to exhibit an increased incidence 
in model violations. Therefore, highly accurate phylogeny 
estimation of large datasets may require the use of new 
methods that are based upon more realistic, and more general, 
models of sequence evolution.

\subsubsection{Bayesian MCMC methods}
Bayesian methods are similar to maximum likelihood methods 
in that the likelihood of a model tree with respect to the 
input sequence alignment is computed during the analysis; 
the main difference is that maximum likelihood selects the 
model tree (both topology and numeric parameters) that 
optimizes the likelihood, while a Bayesian method outputs a 
distribution on trees. However, once the distribution is 
computed, it can be used to compute a single point estimate 
of the true tree using various techniques (e.g., a consensus 
tree can be computed, or the model tree topology with the 
maximum total probability can be returned).

The standard technique used to estimate this distribution 
is a random walk through the model tree space, and the 
distribution is produced after the walk has converged to 
the stationary distribution.

There are many different Bayesian methods (e.g., MrBayes, BEAST, 
PhyloBayes, p4, and BayesPhylogenies), differing in terms 
of the techniques used to perform the random walk, and 
the model under which the likelihood is computed; however, 
MrBayes is the most popular of the methods. Bayesian 
methods provide theoretical advantages compared to 
maximum likelihood methods. However, the proper use of a 
Bayesian MCMC method requires that it run to convergence, 
and this can take a very long time on large datasets. 
Thus, from a purely empirical stand- point, Bayesian 
methods do not yet have the scalability of the best 
maximum likelihood methods, and they are generally 
not used on very large datasets.

\subsubsection{Comparisons between methods}
Simulation studies have shown some interesting 
differences between methods. For example, the 
comparison between neighbor joining and maximum 
parsimony reveals that the relative performance may 
depend on the number of taxa and the rate of evolution, 
with maximum parsimony sometimes performing better 
on large trees with high rates of evolution, even 
though the reverse generally holds for smaller trees.  

More generally, most simulation studies have shown 
that maximum likelihood and Bayesian methods 
(when they can be run properly) outperform maximum 
parsimony and distance-based methods in many 
biologically realistic conditions 
(see Wang et al. for one such study).

\subsubsection{Heuristics for exploring treespace}
Since both maximum likelihood and maximum parsimony 
are NP-hard, methods for ``solving'' these problems 
use heuristics to explore the space of different tree 
topologies. These heuristics differ by the techniques 
they use to score a candidate tree (with the best 
ones typically using information from previous 
trees that have already been scored), and how 
they move within treespace. 
 
% \section{Methods of research}

\newpage