\chapter*{Conclusion}
\addcontentsline{toc}{chapter}{\numberline{}Conclusion}
This chapter set out to survey large-scale phylogeny and alignment estimation. 
As we have seen, large-scale alignment and phylogeny estimation 
(whether of species or of individual genes) is a complicated problem. 
Despite the multitude of methods for each step, the estimation of very 
large sequence alignments or gene trees is still quite diffcult, 
and the estimation of species phylogenies (whether trees or networks!), 
even more so. Furthermore, the challenges in large-scale estimation 
are not computational feasibility (running time and memory), as 
inferential methods can differ in their theoretical and empirical 
performance. 

The aim of this paper is to enable those who have never reconstructed a phylogeny to do so from scratch. The
paper does not attempt to be a comprehensive theoretical guide, but describes one rigorous way of obtaining
phylogenetic trees. Those who follow the methods outlined should be able to understand the basic ideas
behind the steps taken, the meaning of the phylogenetic trees obtained and the scope of questions that can be
answered with phylogenetic methods. The protocols have been successfully tested by volunteers with no
phylogenetic experience.
\newpage