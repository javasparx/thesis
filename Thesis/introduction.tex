\chapter*{Introduction}
 \addcontentsline{toc}{chapter}{\numberline{}Introduction}
\begin{quote}
%Add abstract here
\end{quote}

Phylogenetic is the study of the evolutionary histories of living 
organisms, and represent the evolutionary divergences by finite 
directed (weighted) graphs, or directed (weighted) trees, known 
as \textit{phylogeny}. Based on molecular sequences, phylogenetic trees 
can be built to reconstruct the evolutionary tree of species involved. 
In particular, the representation derived from genes or protein 
sequences is known as \textit{gene phylogeny}, while the representation 
of the evolutionary path of the species are often referred as 
species \textit{phylogeny}. A gene phylogeny is, to some extend, a local desciption. 
It only describes the evolution of a particular gene or encoded protein, 
and this sequence could evolve much more or less differently than 
other genes in the genome, or it may have completely different 
evolutionary history from the rest of the genome due to horizontal 
gene transfers. Therefore, the evolution of a particular gene only 
provides a local picture, not necessarily reflect the \textit{global} 
evolutionary picture of the species. We can only hope that we could 
assemble the jigsaw puzzle pieces with a wide selection of gene 
family to give an overall assessment of the species evolution. 

While in general the topology in phylogenetic trees represents 
the relationships between the taxa, assigning scales to edges 
in the trees could provide extra information on the amount of 
evolution divergence as well as the time of the divergence. 
The phylogenetic trees with the scaled edges are called \textit{phylograms}, 
while the non-scaled phylogenetic trees are called \textit{cladograms}. 
Purely for the sake of computers data processing, some special 
formats were artificially created, such as Newick format. From 
biological point of view, the building of phylogenetic trees 
can be roughly divided into the following steps. 


\textbf{Choose molecular marker.}
In building molecular phylogenetic trees, either nucleotide 
or protein sequence data can be used, but the outcomes from 
the choice could be quite different. The rule of thumb is to 
select nucleotide sequences when some very closely related 
organisms are studied because they tend to evolve more rapidly 
than proteins; and to select protein sequence (or slowly evolving 
nucleotide sequences) when more widely divergent groups of 
organisms are studied.


\textbf{Perform sequence alignment.}
 The sequence alignment establishes positional correspondence 
 in evo- lution, and aligned positions are assumed to be 
 genealogically related. Though there have been numerous 
 so-called \textit{state-of-art} alignment programs available and 
 many times they do help, manual editing is often
 crucial in the quality of alignment, and yet there is 
 no firm or clearly defining rule on how to modify a 
 sequence alignment. 
 
 
 \textbf{Choose a model of evolution.}
  One quantitate measure of divergence between two 
  sequences is the number of substitutions in an alignment. 
  However, this measure can somehow be misleading in 
  representing the true evolution. Not only the nucleotide 
  may actually undergone several intermediate steps of 
  changes behind a single mutation in the sequence, but 
  also an observed identical nucleotide may hide parallel 
  mutations of both sequences. This is known as homoplasy. 
  The statistical models used to correct homoplasy are 
  called evolutioinary models. For constructing DNA 
  phylogenies, there have been Jukes-Cantor model and 
  Kimural model.
  
 
\textbf{Determine a tree building method.}
 There are basically two types of phylogenetic 
 methods, character based methods and distance 
 based methods. Character-based methods based on 
 discrete characters from molecular sequences from 
 individual taxa. The theory is that characters at 
 corresponding positions in a multiple sequence 
 alignment are homologous (this word has different 
 meaning in mathematics and is precisely defined for 
 sequences with differentials.) among the sequences 
 involved. On the other hand, distance-based methods is 
 based on the distance, the degrees of differences between 
 pairs of seqeuences. Such distance will be used to construct 
 the distance matrix between individual pairs of taxa. 
 The theory in this case is that all sequences involved 
 are homologous and the weighted directed tree will 
 satisfy the additive properties. There are two different 
 algorithms in distance based method, the cluster-based 
 and the optimality-based. The cluster-based method 
 algorithms build a phylogenetic tree based on a distance 
 matrix starting from the most similar sequence pairs. 
 The algorithms of cluster-bsed include unweighted pair 
 group method using arithmetic average (UPGMA) and 
 neighbor joining (NJ). The optimality-based method 
 algorithms compare numerous different tree topologies 
 and select the one which is believed to best fit between 
 computed distances in the trees and the desired 
 evolutionary distances which often referred as actual 
 evolutionary distances. Algorithms of optimality based 
 include Fitch-Margoliash and minimum evolution. 

\section*{Motivation}
%\addcontentsline{toc}{section}{\numberline{}Motivation} 

The root of the problem lies in the lack of a certain mechanism 
for analyzing phylogenetic sequences. This makes it possible to 
analyze and build phylogenetic trees from given data. The problem 
is to optimize constructing large trees using estimate 
calculations algorithms.

The aim is to find an optimal solution in a set of candidate solutions. 
There are other types of problems besides optimisation and decision problems. 
This dissertation will deal with optimisation problems.

Problems are not only categorised on the basis of the types of 
answer they seek, but also on the basis of how complex it is to 
compute the answers. For some problems, there is no algorithm 
giving always the correct answer. For most of the problems 
encountered in practice, however, the complexity of their 
solution is measured in terms of the amount of computational 
resources (typically time and memory space) that algorithms need 
in order to compute the correct output.
 

\section*{Research question} 
%\addcontentsline{toc}{section}{\numberline{}Research question}

The following research questions were defined from given below 
background of the problem:

 
\section*{Contributions}
%\addcontentsline{toc}{section}{\numberline{}Contributions}

\section*{Outline}
%\addcontentsline{toc}{section}{\numberline{}Outline}

Although the first section of each chapter gives an overview of
its contents, below outlined the contents of each chapter in a more 
concise way.

In the first chapter phylogeny, trees, gene, gene sequence are disscused.

In the second chapter state of art and how the estimate calculations 
algorithms can be used to solve problem are discussed


