\chapter*{Abstract} 
% \addcontentsline{toc}{chapter}{\numberline{}Abstract} %include page number to ToC%
Bioinformatics has gained increased usage and attention the last years.
In particulary, one of the main part of it philogenetics. 
But, if one wants to analize own research tests in this field, appropriate applications 
are still lagging behind.

This thesis brings Phylogenetics and Informatics together. 
It describes the implementation of a modern algorthms
into the Phylogeny data analizing system. 

In particulary, algorithms for estimate calculations \cite{kamilov} and how they
differ from traditional algorithms discussed.

Aim is to provide a different approach for the 
construction of Phylogeny Trees using different algorithms.\cite{juravlyov}

Both the general and unique features of the system discussed. Also, 
described new features in our analizing system, which were decided 
and added through an evaluation process.

Given examples how to use the analisis system.\par
% \textbf{
% For an MA thesis, the abstract should be between 100 and 250 words, depending on 
% departmental requirements. For the length of a PhD abstract, please consult departmental
% The abstract should normally include the following information: (1) a statement 
% of the problem the research sets out to resolve; (2) the methodology used; (3) the major 
% findings. Other information is optional unless required by the department.
% } 

\vspace{10mm}
% Filogenetik ma'lumotlarni tahlil qilish prosedurasini ishlab chiqish.\par
% Oxirgi yillarda bioinformatika sohasining ko'lami anchagina kengaydi va ommalashdi.
% Xususan, tirik organizmlarning rivojlanishini va o'zgarishlarini o'rganuvchi fan, 
% filogenetika, ham shularga ichiga kiradi. Xozirgi kunning informatika sohasidagi 
% yutuqlarga qaramasdan, filogentik ma'lumotlarni o'rganuvchi izlanuvchilar uchun 
% amaliy dasturiy ta'minotlar sezirarli orqada qolmoqda.

% Bu dissertatsiya ishida zamonaviy algoritmlar yordamida filogenetik ma'lumotlarni
% tahlil qilish usullari izohlangan. Xususan, baholash algoritmlari haqida va ularning 
% boshqa usullardan farqlari haqida so'z boradi.

% Maqsad, turli hil usullardan foydalanib filogenetik ma'lumotlarni tahlil qilish va 
% natijada filogenetik daraxtlar quradigan tizim ishlab chiqishdan iborat. 

% Tizimning asosiy va o'ziga hos imkoniyatlari ko'rsatib o'tilgan.

% Foydalanuvchilar uchun misollar bilan ko'rsatmalar keltirilgan.
\textbf{Uzbekcha}

Hozirda fan taraqqiyoti tadqiq qilinayotgan ob'ektlar yoki ob'ektlar
jamlanmasi tarkibi murakkabligining ortib borishi kuzatilmoqda. 
Ob'ektlar haqidagi axborotning oshib borishi, ushbu axborotni saqlash 
va qayta ishlash uchun kompyuter vositalarining qo'llanilishi, amaliyotda 
to'planagan axborotlar tarkibiga yashiringan qonuniyatlarni aniqlash 
hamda ijobiy hal qilish masalalari yanada dolzarb bo'lib bormoqda. 
Ma'lumotlarni saqlash uchun ma'lumotlar bazasini (MB) yaratish ehtiyoji 
tug'ilib, uning vositasida to'plangan ma'lumotlarga samarali ishlov berish 
yondashuvlari va tizimlarini ishlab chiqish zarur. Bunday mexanizmli 
vositalarni ishlab chiqishda so'ngi vaqtlarda shiddat bilan rivojlanayotgan 
ma'lumotlarni intellektual tahlili usullaridan unumli foydalanish 
maqsadga muvofiq. Ma'lumki, ma'lumotlarni intellektual tahlilining 
asosiy masalalari kassifikatsiya, regressiya, bashoratlash, assotsiativ 
qoidalarni qidirish, klasterizatsiya masalalarini qamrab oladi. 

Biologik tasniflar va ma'lumotlar bazasi yuzlab, minglab va undan 
ham ko'p turli xil darajadagi yuzlab belgilarga ega bo'lgan taksonlardan 
tashkil topadi. Kompyuter aniqlagichlari uchun bunday katta hajmdagi MB 
asosida tasniflashni amalga oshirish juda muhim hisoblanib, bunda qaror 
qabul qilish uchun eng maqbul muqobilni olish talab etiladi.
 
O'simliklar, xayvonlar va zamburug'lar aniqlagichlari 300 
yildan beri biolog olimlar 
tomonidan tadqiq qilib kelmoqda va ularning tuzilish usullari muhokama 
qilinmoqda, ularning biologik tasniflarini sistematik kalitlar 
yordamida dastlabki avtomatizatsiyalash o'tgan asrlardagi hisoblash 
texnikasi yordamida amalga oshirilgan. Tasniflash juda ko'pgina biologik 
tadqiqotlar jarayonida soha olimlarining amaliyotida muhim ahamiyat kasb 
etadi.

O'zbekiston biolog olimlari uchun ham yuqoridagi kabi muammolar vujudga 
kelmoqda. Shuning uchun, tahlillardan kelib 
chiqib qisman pretsendentlikka asoslangan timsollarni 
aniqlash algoritmi asosida biologik tizimlarni tasniflash va tadqiq qilish 
tizimini tashkil etish tadqiqotning dolzarbligini belgilaydi.

\newpage
